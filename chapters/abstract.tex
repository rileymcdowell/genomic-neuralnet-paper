\chapter*{Abstract}

Reduced costs for DNA marker technology coupled has generated a huge amount of
molecular data and greatly increased the options available to characterize lines 
in a breeding program. Concurrently, the field of machine learning under
the moniker of data science has experienced a resurgence of research into 
techniques to detect or "learn" patterns in noisy data in a variety of 
technical applications. Here, we present a review of current genomic prediction 
and machine learning literature. We apply so called "deep learning"
techniques from machine learning research related to regularized neural networks to six 
phenotypic prediction tasks from published reference datasets. 
The neural network models are compared to a selection of regularized bayesian 
and linear regression techniques commonly employed for phenotypic prediction and genomic
selection tasks. Applying regularization frequently improves neural network 
prediction accuracy. On three prediction tasks, regularized neural networks 
are the most accurate model evaluated. The depth of the network architecture
does not appear to influence the accuracy of the trained model. We also find
that concerns about the computer processing time needed to train neural network 
models to perform well in genomic prediction tasks may not apply when Graphics
Processing Units are used for model training.


%The abstract should be written for people who may not read the 
%entire paper, so it must stand on its own.  The impression it makes usually determines 
%whether the reader will go on to read the article, so the abstract must be 
%engaging, clear, and concise.  In addition, the abstract may be the only part of the 
%article that is indexed in databases, so it must accurately reflect the 
%content of the article. A well-written abstract is the  most effective way to reach intended 
%readers, leading to more robust search, retrieval, and usage of the article. 

