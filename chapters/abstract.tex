\chapter*{Abstract}

Reduced costs for DNA marker technology has generated a huge amount of
molecular data and made it economically feasible to generate dense
genome-wide marker maps of lines in a breeding program. to aid in making selections in breeding programs. 
This and greatly increased the options available to characterize lines 
in a breeding program. Research into data science theory and
application has experienced a resurgence of research into 
techniques to detect or "learn" patterns in noisy data in a variety of 
technical applications. Here, we present a review of the genomic prediction 
and machine learning literature. We apply deep learning techniques 
from machine learning research to six phenotypic prediction tasks using 
published reference datasets. Because regularization frequently improves 
neural network prediction accuracy we included regularization methods in the neural network models.
The neural network models are compared to a selection of regularized Bayesian 
and linear regression techniques commonly employed for phenotypic prediction and genomic
selection. On three of the phenotype prediction tasks, regularized neural networks 
were the most accurate of the models evaluated. Surprisingly, for these data sets, the the 
depth of the network architecture did not affect the accuracy of 
the trained model. We also find that concerns about the computer processing 
time needed to train neural network models to perform well in genomic prediction 
tasks may not apply when Graphics Processing Units are used for model training.


%The abstract should be written for people who may not read the 
%entire paper, so it must stand on its own.  The impression it makes usually determines 
%whether the reader will go on to read the article, so the abstract must be 
%engaging, clear, and concise.  In addition, the abstract may be the only part of the 
%article that is indexed in databases, so it must accurately reflect the 
%content of the article. A well-written abstract is the  most effective way to reach intended 
%readers, leading to more robust search, retrieval, and usage of the article. 

