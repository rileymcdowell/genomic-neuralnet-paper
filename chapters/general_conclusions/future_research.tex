\section{Future Research}

\subsection*{Other Models}
Future research should directly compare neural network based genomic prediction
models with other state-of-the-art models. The RKHS regression family of models
has performed well on several prediction tasks \citep{heslot2012, crossa2010, gonzalez-recio2014, gianola2006}.
In order to properly evaluate these models, a flexible model fitting
and evaluation model architecture such as the one built for this paper will be
needed. The Bayesian Alphabet models have also performed notably well on
a wide variety of problems \citep{heslot2012, crossa2010, thavamanikumar2015}.
These also should be included in comparisons with regularized neural networks.
Particular focus must be placed on measurable aspects of the genetic architecture 
of a trait that are correlated with the performance of a particular model. This will 
enable better modeling decisions during the development of a genomic selection program.

\subsection*{Other Regularizers}

This study examined the effects of dropout and L2 normalization on neural networks. It 
did not consider alternative regularization techniques which exist for MLP neural
networks. While LASSO regression utilizes L1 regularization and was evaluated, 
L1 regularization on network weights was not evaluated 
to reduce the total number of network parameters evaluated in this study. 
Still other regularizers exist such as activity and bias regularization which 
penalize the total output of all neurons and the bias terms on each neuron, 
respectively. Both of these forms of regularization 
can be applied in an L1 and L2 sense. Another series of regularizers are more commonly
known as constraints, which constrain network various subsets of network parameters 
within bounds, within norms, or to positive real numbers. Unlike many of the 
linear or Bayesian regression techniques evaluated in this and other studies, 
the family of fully-connected neural networks are amenable to dozens of different 
regularization techniques for thousands of different network architectures. This poses
a combinatorics challenge when attempting to select a single optimal network 
architecture. However, this also creates opportunities to identify networks that 
best model the genetic architecture and thus achieve the highest predictive 
accuracy on a given genomic prediction task.

\subsection*{Additional Input Features}

Neural networks are robust to input feature transformations. It is typical 
to transform inputs to a real number in the range $[0,1]$ prior to using them
as input for a neural network. Molecular marker arrays typically output values that
can be linearly scaled within this range, but other values can be easily included
in addition to marker calls. Several methods could be used to encode non-marker information
to improve predictive accuracy. These include location or population information that can
be encoded using binary features to indicate membership in a particular population
or that an evaluation was conducted in an environment where a pest was present for example.
Non-binary inputs such as weather information on a daily or weekly interval could be 
included and allowed to interact with marker calls to improve phenotypic estimates.
Sophisticated models such as these could be reverse-engineered to estimate optimal
genetic combinations for performance in particular environments. Like other regression
models, the more features that are included in the model, the more regularization is needed
to prevent overfitting and produce predictive outputs. Many other input features
can be conceived to augment the information leveraged by a network when making
a prediction. While neural networks pose the aforementioned combinatoric architecture
challenge, they are unique in accepting large numbers of inputs from diverse 
sources of data. Practically any information can be transformed to 
a $[0,1]$ scale and leveraged by a network. In this way, networks can 
form so-called expert systems that perform in ways that are similar to 
that of a human expert.

\subsection*{Larger Datasets}

This study examined both small and medium density marker datasets. Datasets with
many thousands of marker calls were not included due to time and budgetary constraints.
Future genomic prediction research should include larger datasets with $p \gg n$. Several
examples of these datasets are available in the literature \citep{resende2012, cleveland2012}.

\subsection*{Radial Basis Function Networks}

This work has focused on deep MLP neural networks with sigmoid activation functions.
Architectures such as the Radial Basis Function (RBF) network do not possess the multiple hidden
layers of MLP networks. This alternative architecture is not well studied in the machine learning 
literature, but has shown promising performance on genomic selection 
problems \citep{gonzalez-camacho2012}.

\subsection*{Final Remarks}

Additional research is needed to determine the performance of 
neural networks relative to other state-of-the-art models. This process can
be accelerated by leveraging GPU computation to more quickly evaluate
large numbers of complex networks. As transistor density continues to
increase \citep{moore1965}, and GPU computational capacity subsequently
increases, a network's computational demands will decrease thus making them more
attractive for use in genomic prediction tasks as well as in other domains.

