\section{Introduction} \label{sec:gen-intro}

Humans have been breeding plants for food since the dawn of agriculture. This process
emerged from a haphazard form of artificial selection where farmers would save seed from the 
highest yielding, healthiest, and easiest to harvest plants to plant the following year. 
Over thousands of years, this process produced many of the plants of  
agricultural value today.

Between 1930 and 1960, academic agricultural and plant breeding communities were setting the
stage for agricultural giants such as Norman Borlaug, Henry A. Wallace. These individuals 
explored options for breeding plants for specific traits and applying rigorous, 
scientifically-motivated management practices to improve varieties and the total yield 
of agriculture worldwide.

As an example, in 1930, corn was an open-pollinated crop with an average 
yield around 20-30 bu/ac. By 1960, field corn had transitioned to become a single-cross 
hybrid crop producing twice the yield of those only two or three decades earlier. The implementation
of agricultural practices that would drive the green revolution had been kicked off, and
the stage was set for a rapid improvement in yield and genetic gain in
corn as well as many other crops which continued into the 2000s \citep{evenson2003}.

The green revolution drove crop production yields higher through a combination of 
improved agricultural practices and varieties that had greater yield potential
and agronomic stability. By the year 2000, advances in agronomy were slowing, and 
molecular breeding technologies were promising the next advancement in crop yield
through a focus on a better understanding of crop genetics and genomics. 
The advent of Bt Corn was the landmark technology delivering on the promise of 
biotechnology in agriculture. Bt corn was first commercialized in in 1996 and 
accounted for 63\% of U.S. grain corn by 2010 \citep{fernandez2012}. 

Biotechnology in agriculture is not limited only to transgenic traits. The cost of 
DNA sequencing and genetic marker assays began to fall as the green revolution
came to a close. The cost of SNP assays has dropped so sharply that today it is 
possible to apply them to many thousands of samples in large breeding programs \citep{hiremath2012}. 
In many cases, it is now less costly to genotype a seed than to grow it and observe its phenotype.

Because a whole genome SNP assay can serve as an approximation for the genome sequence, 
it is possible to a SNP haplotype to predict the genetic
potential of a plant in a affordable way. This process is known 
as "genome-enabled prediction" hereafter called genomic prediction, and when used to inform breeding 
program selections is known as genomic selection. Genomic prediction is typically 
applied early in a breeding program as a way to increase the overall selection 
pressure in a breeding program and thus increase the rate of genetic gain. A review of the 
literature and methods to perform this technique is presented in Section \ref{sec:lit-review}. 

The genetic gain in a breeding program is directly related to many factors. A 
general guideline to expected genetic gain each year in an inbred breeding 
program is presented in Equation \ref{eq:genetic_gain} \citep{fehr1987}. The equation is
often used as a general guideline to explore whether changes in breeding strategy will
have a positive or negative effect on genetic gain. 

\begin{equation} \label{eq:genetic_gain}
\begin{split}
    G_y           &= \textrm{genetic gain per year} \\
    k             &= \textrm{selection differential} \\
    r             &= \textrm{number of replications} \\
    t             &= \textrm{number of environments} \\
    n             &= \textrm{plants per plot} \\
    \sigma^2_{A}  &= \textrm{additive genetic variation} \\
    \sigma^2_{u}  &= \textrm{within-plot environmental variation} \\
    \sigma^2_{wg} &= \textrm{within-plot genetic variation} \\
    \sigma^2      &= \textrm{between-plot variation} \\
    \sigma^2_{ge} &= \textrm{genotype by environment interaction variation} \\
    \sigma^2_{g}  &= \textrm{genotypic variation} \\
    G_y           &= \frac{k\sigma^2_A}{y \sqrt{ \frac{\frac{\sigma^2_{u} + \sigma^2_{wg}}{n} + \sigma^2}{rt} + \frac{\sigma^2_{ge}}{t} + \sigma^2_{g} }} \\
\end{split}
\end{equation}

From Equation \ref{eq:genetic_gain}, it is clear that improvements in genetic gain per year can
be made by increasing the number of replications and/or environments tested, increasing the 
number of plants within each experimental plot, or increasing overall selection pressure 
during the breeding program. Genomic selection improves genetic gain by addressing 
all of these facets simultaneously.

\begin{itemize}
    \item A breeder may grow many more plants than can be evaluated using field plots in 
          early generations. This allows a program to increase the overall selection 
          differential without reducing the number of new lines at the end of the breeding pipeline.
    \item Because genomic prediction estimates genetic potential directly from genetic information,
          genotype by environment interaction is reduced to zero.
    \item When data collection is properly calibrated and quality controlled typically only one
          SNP assay is required per sample. This results in near zero measurement error of the 
          genetic information of a plant. This reduces within-plot and between-plot 
          (or more accurately, within genotype and between genotype) variation from all sources to zero.
\end{itemize}

However, unlike in the genetic gain equation, genomic prediction introduces prediction error to 
the denominator of the equation. When practicing phenotypic selection, measurement errors 
directly reduce the rate of genetic gain. When performing genomic prediction, measurement errors 
affect the training data that the statistical model uses to make predictions, reducing its predictive 
accuracy. This error can be minimized by by taking accurate measurement like with phenotypic selection, 
but also benefits from exposure to data from related populations or previous testing years. 
By comparing the magnitude of prediction error with the cost of executing early-generation 
SNP assays in a breeding program, it is possible to determine if genomic selection is 
beneficial to a breeding program. Reducing the total genomic prediction error thus has a 
direct positive impact on genetic gain. This thesis explores the application of regularized 
neural networks to minimizing genomic selection prediction error.

