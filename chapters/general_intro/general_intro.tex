\section{Introduction}

Humans have been breeding plants for food since the dawn of agriculture. This process
emerged from a haphazard form of artifical selection where farmers would save seed from the 
highest yielding, healthiest, and easiest to harvest plants to harvest the following year. 
Over thousands of years, this process produced many of the subspecies of plants we consume today. 

% Wikipedia Green revolution article.
Between 1930 and 1960, the world underwent the 'Green Revolution', when agricultural giants such
as Norman Borlaug, Henry A. Wallace, and many others explored options to breed plants for
specific traits, and apply rigorous, scientifically-motivated management practices to improve
varieties and agruculture practice worldwide.

% Pioner published "Developing a Superior Maize Hybrid"
% https://www.pioneer.com/CMRoot/Pioneer/About_Global/news_media/media_library/articles/maize_hybrid.pdf
By 1960, corn had transitioned from an open-pollitaed crop with an average yield around 20-30 bu/ac
to a crop consisting primarily of single-cross hybrids producing twice the yield of those bred 
only two or three decades earlier. The improvement in yield and genetic gain year-over-year in
breeding programs has yet to slow significantly since rigorous hybrid breeding became 
commonplace after 1960.

% Common knowledge.
More recently, the monetary cost of using molecular techniques to measure and even modify plant
genetics artificially has dropped significantly, ushering in an opportunity to inprove 
the rate of genetic gain in plant breeding once again.

Prior to the tex 

% Fehr's Principles of Cultivar Development.
The genetic gain in a breeding program is directly related to a host of factors. A 
general guideline to expected genetic gain each year in a generic inbred breeding 
program is presented in equation \ref{eq:genetic_gain} \citep{fehr1987}.

\begin{equation} \label{eq:genetic_gain}
\begin{split}
    G_y           &= \textrm{genetic gain per year} \\
    k             &= \textrm{selection differential} \\
    r             &= \textrm{number of replications} \\
    t             &= \textrm{number of environments} \\
    n             &= \textrm{plants per plot} \\
    \sigma^2_{A}  &= \textrm{additive genetic variation} \\
    \sigma^2_{u}  &= \textrm{within-plot environmental variation} \\
    \sigma^2_{wg} &= \textrm{within-plot genetic variation} \\
    \sigma^2      &= \textrm{between-plot variation} \\
    \sigma^2_{ge} &= \textrm{genotype by environment interaction variation} \\
    \sigma^2_{g}  &= \textrm{genotypic variation} \\
    G_y           &= \frac{k\sigma^2_A}{y \sqrt{ \frac{\frac{\sigma^2_{u} + \sigma^2_{wg}}{n} + \sigma^2}{rt} + \frac{\sigma^2_{ge}}{t} + \sigma^2_{g} }} \\
\end{split}
\end{equation}

The goal of a breeding program is to maximize the total genetic gain per breeding cycle, 
which consists of multiple years of breeding. Controllable ways to improve genetic gain 
per cycle involve increasing the number of replications iand/or environments tested,
increasing the number of plants within each experimental plot, or increasing overall selection pressure
during the breeding year.

Genomic selection allows breeders to grow large cohorts of plants which would ordinarily be 
tested in one location with one or two replications to establish a genetic value and instead 
estimate the genetic value based on previously establised genetic effects. 


This is a section of text about genomic selection. 

Let's cite Gianola here \citep{gianola2006}.

\blindtext

% Modeling Approaches?

\blindenumerate
\blindtext
