\documentclass[9pt,twocolumn,twoside]{g3_article/gsag3jnl}

% Packages.

\usepackage{multirow}
\usepackage{caption}
\usepackage{tikz}

% Custom Commands.
\newcommand{\etal}{\textit{et al.}}

\articletype{gs} % article type
% {inv} Investigations
% {msr} Mutant Screen Reports
% {gs} Genomic Selection
% {goi} Genetics of Immunity 
% {gos} Genetics of Sex 
% {mp} Multiparental Populations

\title{Genomic Selection with Deep Neural Networks}

\author[$\ast$,1]{Riley McDowell}
\author[$\dagger$]{David Grant}
%\author[$\ddagger$]{Author Three}
%\author[$\S$]{Author Four}
%\author[$\ast\ast$]{Author Five}

\affil[$\ast$]{Iowa State University, US}
\affil[$\dagger$]{Iowa State University, US}
%\affil[$\ddagger$]{Author three affiliation}
%\affil[$\S$]{Author four affiliation}
%\affil[$\ast\ast$]{Author five affiliation}

%For the authors' names, indicate different affiliations with the 
%symbols: $\ast$, $\dagger$, $\ddagger$, $\S$. After four authors
%, the symbols double, triple, quadruple, and so forth as required.

\keywords{ genomic prediction \\ deep learning \\ neural network \\ genomic selection \\ SNP }

\runningtitle{ Genomic Selection with Deep Neural Networks } % Goes into the footer.

\correspondingauthor{Corresponding Author HERE} % TODO: Who is this?

\begin{abstract}

Reduced costs for DNA marker technology coupled has generated a huge amount of
molecular data and greatly increased the options available to characterize lines 
in a breeding program. Concurrently, the field of machine learning has experienced
a resurgence of research into techniques to detect or "learn" patterns in noisy
data in a variety of technical applications. Here, we apply so called "deep learning"
techniques from current machine learning research related to neural networks to five 
genomic selection and phenotypic selection problems using published reference datasets. 
We compare the results of these algorithms to a selection of bayesian and linear 
regression techniques commonly employed today. TODO: FINDINGS/RESULTS/CONCLUSIONS.


%The abstract should be written for people who may not read the 
%entire paper, so it must stand on its own.  The impression it makes usually determines 
%whether the reader will go on to read the article, so the abstract must be 
%engaging, clear, and concise.  In addition, the abstract may be the only part of the 
%article that is indexed in databases, so it must accurately reflect the 
%content of the article. A well-written abstract is the  most effective way to reach intended 
%readers, leading to more robust search, retrieval, and usage of the article. 
%
%Please see additional guidelines notes on preparing your abstract below.

%\begin{itemize}
%\item provide a synopsis of the entire article;
%\item begin with the broad context of the study, followed by specific background for the study;
%\item describe the purpose, methods and procedures, core findings and results, and conclusions of the study;
%\item emphasize new or important aspects of the research;
%\item engage the broad readership of G3 and be understandable to a diverse audience (avoid using jargon);
%\item be a single paragraph of less than 250 words;
%\item contain the full name of the organism studied;
%\item NOT contain citations or abbreviations.
%\end{itemize}

\end{abstract}

\setboolean{displaycopyright}{true}

\begin{document}

\maketitle
\thispagestyle{firststyle}
\logomark
\articletypemark
\marginmark
\firstpagefootnote
\correspondingauthoraffiliation{TODO: CORRESPONDING AUTHOR ADDRESS AND EMAIL HERE}
\vspace{-11pt}

\noindent % Skip header for 'Introduction'.

The wide availability and reduced cost of molecular marker technology
has created opportunities to perform marker assisted selection of genotypes
in plant and animal breeding. Quantitative Trait Locus (QTL) mapping techniques
have proved useful for detecting the effects of small numbers of markers producing 
phenotypes with large effects \citep{miles2008}. Once the QTL 
markers have been identified, they can be leveraged for making selections
on a population. However, when making selections on traits affected by many 
alleles with varying effect sizes distributed widely across a genome, 
QTL mapping is less effective. 

Unlike QTL mapping, genomic selection methods attempt to predict phenotypes 
utilizing all available SNP marker data collected from a population,
allowing one of many possible statistical models to learn the marker-trait 
associations in a data driven way \citep{meuwissen2001}. This techique has
proven effective but relies on an appropriate choice of the statistical 
model that will most accurately predict phenotype from a panel of high-density 
marker data. It is likely that the best statistical model for 
genomic selection is dependent on the genetic archetecture of the 
predicted trait \citep{crossa2010, gonzalez-camacho2012, 
resende2012, cleveland2012, thavamanikumar2015}.  Mathematically, models incorporating 
interactions between marker features have the capacity to achieve higher accuracy 
by caputuring non-additive effects. Experimental results support this 
hypothesis \citep{gonzalez-camacho2012}. Alternative prediction methods continue 
to be an active area of research in plant and animal breeding \citep{koning2012}.

Concurrent with the advent of genomic selection as a practice the popularity of the 
interdisciplinary field known as data science has increased. Practitioners of data 
science apply machine learning and statistics to make predictions, usually
by applying ideas or techniques from a wide variety of domains 
including mathematics, physics, and computer science. Often, a data scientist's focus is to
create a predictive model than may not be associated with an underlying generative model. 
This can be viewed as the distinction between data science and classical statistics 
\citep{donoho2015, breiman2001}. The rapid increase in popularity of data science
is associated with an increase in definition of best practices for predictive modeling
across many disciplines as well as software packages to automate the 
process of building predictive models from any data source.

Neural networks are a type of model frequently employed by data scientists
for predictive modeling. Neural networks consist of layers of interconnected neurons
which map inputs to one or more outputs. Each neuron in a network can be expressed as a 
transformation of a weighted sum of $n$ inputs 

\begin{equation}
    output_{lk} = \sum_{i=1}^{n} f_l(w_{lki} * x_{i} + b_{lk})
\label{eq:neuron}
\end{equation}

where $output_{lk}$ is the output from neuron $k$ in network layer $l$ having activation
function $f_l$, weights $w_{lki}$ and bias $b_{lk}$.

A neural network is a collection of neurons that map a 
length $n$ input vector $x = (x_1, ..., x_n)$ through a series of $j$ 
"hidden" layers $(l_1, ..., l_j)$. Each hidden layer consists of a variable 
number of neurons, each of which apply an associated coefficient, bias, and 
mathematical transformation to their input and forward the 
result on to every neuron in the subsequent layer forming a network (Figure \ref{fig:deepnet}).

\begin{figure}[htbp]
\renewcommand{\familydefault}{\sfdefault}\normalfont
\centering
% A deep neural network with 3 hidden layers of varying sizes.

\def\layersep{1.5cm}

\begin{tikzpicture}[shorten >=1pt, ->, draw=black!50, node distance=\layersep]
    \tikzstyle{every pin edge}=[<-, shorten <=1pt]

    \tikzstyle{neuron}=[circle, fill=black!25,minimum size=17pt, inner sep=0pt]
    \tikzstyle{input neuron}=[neuron];
    \tikzstyle{output neuron}=[neuron];
    \tikzstyle{hidden neuron}=[neuron];

    \tikzstyle{annot}=[text width=4em, text centered]

    \foreach \name / \y in {1,...,5}
        \node[input neuron, pin=left:Marker \#\y] (I-\name) at (0,-\y) {};

    \foreach \name / \y in {1,...,4}
        \path[yshift=-0.5cm]
            node[hidden neuron] (H1-\name) at (\layersep * 1,-\y cm) {};

    \foreach \name / \y in {1,...,5}
        \path[yshift=0.0cm]
            node[hidden neuron] (H2-\name) at (\layersep * 2,-\y cm) {};

    \node[output neuron, pin={[pin edge={->}]right:Prediction}, right of=H2-3] (0) {};

    \foreach \source in {1,...,5}
        \foreach \dest in {1,...,4}
            \path (I-\source) edge (H1-\dest);

    \foreach \source in {1,...,4}
        \foreach \dest in {1,...,5}
            \path (H1-\source) edge (H2-\dest);

    \foreach \source in {1,...,5}
        \path (H2-\source) edge (0);

    \node[annot, above of=I-1, node distance=1cm] (il) {Input Layer};
    \node[annot, right of=il] (hl1) {Hidden Layer 1};
    \node[annot, right of=hl1] (hl2) {Hidden Layer 2};
    \node[annot, right of=hl2] {Output Layer};
\end{tikzpicture}


\caption{A deep feedforward neural network. Many input marker calls are mapped 
to one or more sequential hidden layers of neurons. For genomic selection, the
input layer often consists of one neuron per marker and the output consists of a single
neuron which combines the information from the final hidden layer to predict a phenotype or BLUP.
The presence of more than one hidden layer indicates that a network is likely to learn
higher order non-linear interactions, and can be called a deep network.}
\label{fig:deepnet}
\end{figure}

Once the network is defined, it must be exposed to input and desired output
values, and adjusted to minimize error in output in a process known as training.
Error in the output of the network is propagated back through the hidden 
layers, and the weights and biases are updated in the direction that would 
decrease output error on many sets or subsets of the input data. 
This turns the network training process into general 
function minimization problem, where the parameters to the function are the 
weights and biases of the network neurons and the function to be 
minimized is the squared differences between the network outputs and 
the desired outputs. The process of propagating output error back 
through a neural network is known as backpropagation, and has been used 
and improved extensively since its description in the 1980s \citep{rumelhart1986}. 

%Good post, good links to the 'deep' part of deep nets.
%http://stats.stackexchange.com/questions/182734/what-is-the-difference-between-a-neural-network-and-a-deep-neural-network

Networks with many hidden layers are notoriously difficult to train due to
the vanishing gradient problem \citep{hochreiter1998}. Recently, a series 
of breakthroughs in neural network training have allowed efficient 
training of deeper networks than were previously possible \citep{sutskever2013}.
A history of the deep learning literature is available in \cite{lecun2015}.
The increased training efficiency and potential to capture relationships 
between input features drove a need to differentiate these deeper networks
from prior work, resulting in the emergence of the phrase "deep learning" 
to describe the construction and training of deep neural networks.

Attempts to apply neural networks to genomic selection has resulted in overfitting of the 
network to training data as well as concerns over the 
computation complexity required to fit the model to datasets with many
markers across many genotypes \citep{heslot2012, gonzalez-recio2014}. 
These results are not surprising. Multi-layer feedforward neural networks 
are capable of approximating functions of arbitrary complexity to arbitrary 
accuracy if provided enough neurons in even a single hidden 
layer. This property of neural networks is known as the 
universal approximation theorem, and can result in 
overfitting if the weights of the network are not regularized in some way \citep{hornik1989}.

Given the promising results from regularized and bayesian methods for 
genomic selection such as ridge or lasso regression and the bayesian family of regressors,
it is prudent to evaluate some of the many of neural network training algorithms which
incorporate regularization of weights during training. Today, networks based on these 
and other regularization techniques continue to achieve award-winning performance 
across many domains \citep{schmidhuber2015}. Similarly, while neural networks are 
computationally demanding to train, the training algorithms 
themselves are often easily expressed with vector and matrix algebra and thus amenable to
Graphics Processing Unit (GPU) accelerated computing. Some report up to sixty-fold speedups 
in training time \citep{sierra2010, schmidhuber2015}. 

In this paper, we present the results of applying two types of regularization
to deep neural networks. The first is a weight decay regularization
which penalizes the weights $W$ with very large values 
similar to ridge regression \citep{krogh1992}. The second is known as dropout 
regularization, where subsets of neurons and their connections are removed 
at random during training, encouraging subsets of the network to learn 
to recognize input features independently. This allows neurons to adapt 
and build independently operating units and prevents the 
neurons from co-adapting, reducing overfitting \citep{srivastava2014}.  

% For the introduction, authors should be mindful of the broad readership of the journal. i
% The introduction should set the stage for the importance of the work to a generalist 
% reader and draw the reader in to the specific study. The scope and impact of the 
% work should be clearly stated.

\section*{Materials and Methods}

Five benchmark datasets of paired phenotypic and genotypic marker data were predicted 
using a collection of regression techniques. Benchmark species included three food crops, 
one forestry species, and one animal species. Benchmark traits include a variety of high and low
heritability traits with simple and complex genetic archetectures. Each dataset was divided
into 10 partitions by drawing entries without replacement and predicted with each
regression technique using ten-fold cross validation. The partitions of the data were 
held constant across all regression techniques to facilitate a fair comparision of the
prediction accuracy of each technique.

For statistical models with tunable hyperparameters, a grid-search of the parameter 
space was conducted and the parameters that produced maximum accuracy on the hold-out
data folds are used in the final analysis. 

The full analysis pipeline, including formatted datasets, quality control processes, 
implementations of each model, and final hyperparameter settings are available in File S1 *TODO* and
in a public repository on github \citep{mcdowell2016}.

%Manuscripts submitted to G3 should contain a clear description of the experimental design in sufficient detail so that 
%the experimental analysis could be repeated by another scientist. If the level of detail necessary to explain the 
%protocol goes beyond two paragraphs, give a short description in the main body of the paper and prepare a detailed 
%description for supporting information.  For example, details would include indicating how many individuals were used, 
%and if applicable how individuals or groups were combined for analysis. If working with mutants indicate how 
%many independent mutants were isolated. If working with populations indicate how samples were collected 
%and whether they were random with respect to the target population.

\subsection*{Statistical Analysis} 

Least squares, ridge, lasso, and elastic net regression were applied as examples of
linear regression with and without normalization penalties. Bayesian 
ridge regression is a bayesian version of ridge regression that assumes that 
the residual error in the model is gaussian distributed. These five regression 
techniques are implemented in the scikit-learn python package version 0.17.1 \citep{scikit-learn}.

A single parameterized neural network with dropout and weight-decay parameters was built using 
the Keras modular neural network library \citep{chollet2015}. The neural network results presented in this
manuscript were trained and evaluated using the default Theano backend \citep{team2016}. 
All training was conducted on an NVIDIA GTX 680 graphics card with 4GB 
of RAM using the CUDA 7.5 toolkit \citep{nickolls2008}. A network with no regularization,
as well as with one or both of the weight-decay and dropout regularization parameters
supplied was trained on each dataset. A summary of all regression methods is presented in Table \ref{tab:regression-methods}.

\begin{table*}[htbp]
\renewcommand{\familydefault}{\sfdefault}\normalfont
\centering
\caption{\bf Regression Methods}
\begin{tableminipage}{\textwidth}
% Benchmark Datasets table.

\begin{tabularx}{\textwidth}{ X m{5em} m{10em} }
\hline
\header Method & Abbreviation & Library \\
\hline
Ordinary Least Squares Regression            & OLS   & scikit-learn   \\
Ridge Regression                             & RR    & scikit-learn   \\
LASSO Regression                             & LASSO & scikit-learn   \\
Elastic Net Regression                       & EN    & scikit-learn   \\ 
Baysian Ridge Regression                     & BRR   & scikit-learn   \\
Unregrularized Neural Network                & N     & Keras (Theano) \\
Neural Network with Weight Decay             & NWD   & Keras (Theano) \\
Neural Network with Dropout                  & NDO   & Keras (Theano) \\
Neural Network with Weight Decay and Dropout & NWDDO & Keras (Theano) \\
\hline
\end{tabularx}


\label{tab:regression-methods}
\footnotesize  
\end{tableminipage}
\end{table*}

Each regression technique was trained and evaluated on each phenotypic trait in all 10 cross-validation folds of each dataset.
The correlation of the predicted and actual phenotypic values was taken for each fold, and the average and 
standard deviation of the resulting correlation coefficients for each regression technique were reported.
 
%Indicate which statistical analysis has been performed and describe the method and model applied. 
%If many genes were examined simultaneously, or many phenotypes, a multiple comparison correction should be 
%used to control the type I error rate, or a rationale for not applying a correction must be provided. 
%The type of correction applied should be clearly stated. It should also be clear whether the p-values 
%reported are raw, or after correction. Corrected p-values are often appropriate, but raw p-values 
%should be available in the supporting materials so that others may perform their own corrections. 

\subsection*{Benchmark Datasets}

Arabidopsis, loblolly pine, maize, pig, and wheat datasets were collected from the author's web pages
or the supplementary information published with their respective papers \citep{loudet2002, resende2012, crossa2010, cleveland2012, thavamanikumar2015}.
Species, authorship, marker, and sample information is summarized in Table \ref{tab:benchmark-datasets}.

\begin{table*}[htbp]
\renewcommand{\familydefault}{\sfdefault}\normalfont
\centering
\caption{\bf Benchmark Datasets}
\begin{tableminipage}{\textwidth}
% Benchmark Datasets table.

\begin{table*}[htbp]
\renewcommand{\familydefault}{\sfdefault}\normalfont
\centering
\caption{\bf Benchmark Datasets}
\begin{tableminipage}{\textwidth}


\newcommand{\etal} {\textit{et al.}}

\newcommand{\loudet}        {\multirow{2}{*}{\parbox{4.0em}{Loudet \etal ~2002}}}
%\newcommand{\resende}       {\multirow{2}{*}{\parbox{4.5em}{Resende \etal ~2012}}}
\newcommand{\crossa}        {\multirow{2}{*}{\parbox{4.0em}{Crossa \etal ~2010}}}
%\newcommand{\cleveland}     {\multirow{2}{*}{\parbox{4.5em}{Cleveland \etal ~2012}}}
\newcommand{\thavamanikumar}{\multirow{2}{*}{\parbox{4.5em}{Thavamanikumar \etal ~2015}}}
\newcommand{\phenom}        {\multirow{2}{*}{\parbox{4.0em}{Measure-ment}}}
\newcommand{\blupm}         {\multirow{2}{*}{BLUP}}


\newcommand{\arabidopsis}   {\multirow{2}{*}{Arabidopsis}}
\newcommand{\loblolly}      {\multirow{2}{*}{Loblolly Pine}}
\newcommand{\maize}         {\multirow{2}{*}{Maize}}
\newcommand{\pig}           {\multirow{2}{*}{Pig}}
\newcommand{\wheat}         {\multirow{2}{*}{Wheat}}


\begin{tabularx}{\textwidth}{ m{7em} m{5em} m{10em} m{4em} m{4em} m{5em} }
\hline
    \header Author & Species & Trait & Markers\protect\footnotemark[1]  & Samples\protect\footnotemark[2] & Dependent Variable \\
\hline
\loudet         & \arabidopsis & Days to flowering w/ short day length     & 69     & 415   & \phenom \\
                &              & Low resource dry matter accumulation      & 69     & 415   &         \\
%\hline
%\resende        & \loblolly    & Crown width age 6 yrs                     & 4,700  & 861   & \phenom \\
%                &              & Wood lignin age 4 yrs                     & 4,698  & 910   &         \\
\hline
\crossa         & \maize       & Well-watered grain yield                  & 1,135  & 264   & \phenom \\
                &              & Female flowering date                     & 1,148  & 284   &         \\
%\hline
%\cleveland      & \pig         & Trait 1 (low heritabilty)                 & 52,843 & 2,804 & \phenom \\
%                &              & Trait 5 (high heritability)               & 52,843 & 3,184 &         \\
\hline
\thavamanikumar & \wheat       & Time to young microspore                  & 797    & 324   & \blupm  \\
                &              & Spike grain number                        & 797    & 324   &         \\
\hline
\end{tabularx}

\label{tab:benchmark-datasets}
\footnotesize  
\raggedright

\footnotemark[1] Markers with greater than 20\% missing calls were filtered from analysis.
\\
\footnotemark[2] Samples with missing phenotypic measurements or greater than 50\% missing marker calls were filtered from analysis.

\end{tableminipage}
\end{table*}

\label{tab:benchmark-datasets}
\footnotesize  
\end{tableminipage}
\end{table*}

Question for Dr. Grant when Intro + M&M draft is ready.

One thing that is important to me is to apply consistent QC and analysis to each dataset. 
I feel that giving unique treatment to each dataset will make for a clunky and difficult 
to read paper. I prefer a generic approach that would be more likely to apply to other 
datasets or readers who might want to try this on their own data. In some cases, 
this will mean that what I do is different from what was done with the data in 
the original publication. For instance, Cleveland uses phenotypic measurements to 
derive breeding values, then uses genomic selection to estimate them back using 
cross-validation. However, Cleveland et al. published the phenotypic measurments, 
and I would rather predict phenotypic measurements when possible like in Loudet, 
Resende, and Crossa. Similarly, Thavamanikumar only published the BLUPs (not the raw data) 
for two different populations, then, among other things, tried to predict BLUPs 
in one population using the data in the other one. I plan to lump the Thavamanikumar 
data together into one population and make predictions using only markers that are 
polymorphic within and across both populations. I arrived at this plan because 
it makes for a highly consistent analysis that is identical in all datasets.

Questions: Do you feel this approach is reasonable? Given that this is primarily a methods 
paper, do I need to detail how each of these is slightly different than the original 
authors' methods in the paper itself or can I rely on my common quality control and 
analysis description to illustrate the differences.

Marker calls were scaled to the range $[-1, 1]$ for all SNP information. If more than 20\% of
marker calls were missing for a sample, the sample was discarded. If fewer than 20\% were missing,
the average value for that marker was imputed for the missing values with one exception: if data were 
published with a marker imputation technique already applied, no further imputation was attempted.

Individuals without phenotypic measurements were discarded from further analysis. A combination of phenotypic 
measurements and deregressed breeding values were predicted, depending on which was provided by the authors.

In addition to their original publications, copies of each of the benchmark datasets as well 
as modified versions formatted for compatibility with the analyses presented in 
this paper are available on github \citep{mcdowell2016}. File S1 *TODO* contains an shapshot of the 
source code and data files that were used to generate the results presented in this publication.

%At the end of the Materials and Methods section, include a statement on reagent and data availability. 
%Please read the Data and Reagent Policy before writing the statement. Make sure to list the 
%accession numbers or DOIs of any data you have placed in public repositories. List the file 
%names and descriptions of any data you will upload as supplemental information. 
%The statement should also include any applicable IRB numbers. You may include specifications for 
%how to properly acknowledge or cite the data.
%
%For example: Strains are available upon request. File S1 contains detailed descriptions of all supplemental files. 
%File S2 contains SNP ID numbers and locations. File S3 contains genotypes for each individual. 
%Sequence data are available at GenBank and the accession numbers are listed in File S3. 
%Gene expression data are available at GEO with the accession number: GDS1234. 
%Code used to generate the simulated data is provided in file S4. 

\section*{Results and Discussion}

The results and discussion should not be repetitive. The results section should give a factual 
presentation of the data and all tables and figures should be referenced; the discussion 
should not summarize the results but provide an interpretation of the results, and should 
clearly delineate between the findings of the particular study and the possible impact 
of those findings in a larger context. Authors are encouraged to cite recent work 
relevant to their interpretations. Present and discuss results only once, not in 
both the Results and Discussion sections. It is sometimes acceptable to combine 
results and discussion. The text should be as succinct as possible. 
Heed Strunk and White's dictum: "Omit needless words!"

%\section*{Additional guidelines}
%
%    \subsection*{Numbers} In the text, write out numbers nine or less except as part of a date, a fraction or decimal, 
%                          a percentage, or a unit of measurement. Use Arabic numbers for those larger than nine, 
%                          except as the first word of a sentence; however, try to avoid starting a sentence with such a number.
%
%    \subsection*{Units} Use abbreviations of the customary units of measurement only when they are preceded by a number: 
%            "3 min" but "several minutes". Write "percent" as one word, except when used with a number: 
%            "several percent" but "75\%." To indicate temperature in centigrade, use ° 
%            (for example, 37°); include a letter after the degree symbol only when some 
%            other scale is intended (for example, 45°K).
%
%    \subsection*{Nomenclature and Italicization} Italicize names of organisms even when  when the species is 
%        not indicated.  Italicize the first three letters of the names of restriction enzyme cleavage 
%        sites, as in HindIII. Write the names of strains in roman except when incorporating 
%        specific genotypic designations. Italicize genotype names and symbols, including all components 
%        of alleles, but not when the name of a gene is the same as the name of 
%        an enzyme. Do not use "+" to indicate wild type. Carefully distinguish between genotype 
%        (italicized) and phenotype (not italicized) in both the writing and the symbolism.
%
%\section*{In-text Citations}
%
%Add citations using the \verb|\citep{}| command, for example \citep{neher2013genealogies} or for multiple citations, \citep{neher2013genealogies, rodelsperger2014characterization}.
%
%For examples of different references, please see the example bibliography file 
%(accessible via the Project menu in the Overleaf editor). This contains examples 
%of articles \citep{neher2013genealogies, rodelsperger2014characterization}, a 
%book \citep{Sturtevent2001}, a book 
%chapter 
%XXXX-SKIPPED-XXXX
%%\citep{Sturtevent2001chp7}
%, ahead-of-print work \citep{Starita2015} and software \citep{Kruijer2015}.
%
%\section*{Examples of Article Components}
%\label{sec:examples}
%
%The sections below show examples of different header levels, which you can use in the primary sections of the manuscript (Results, Discussion, etc.) to organize your content.
%
%\section*{First level section header}
%
%Use this level to group two or more closely related headings in a long article.
%
%\subsection*{Second level section header}
%
%Second level section text.
%
%\subsubsection*{Third level section header:}
%
%Third level section text. These headings may be numbered, but only when the numbers must be cited in the text. 
%
%\section*{Figures and Tables}
%
%Figures and Tables should be labelled and referenced in the standard way using the \verb|\label{}| and \verb|\ref{}| commands.
%
%\subsection*{Sample Figure}
%
%Figure \ref{fig:spectrum} shows an example figure.
%
%\begin{figure}[htbp]
%\renewcommand{\familydefault}{\sfdefault}\normalfont
%\centering
%\includegraphics[width=\linewidth]{images/example-figure-g3}
%\caption{Example figure from \url{http://dx.doi.org/10.1534/g3.115.017509}. Please include your figures in the 
%    manuscript for the review process. You can upload figures to Overleaf via the Project menu. Upon acceptance, 
%    we'll ask for your figure files to be uploaded in any of the following formats: TIFF (.tiff), JPEG (.jpg), 
%    Microsoft PowerPoint (.ppt), EPS (.eps), or Adobe Illustrator (.ai).  Images should be a minimum of 
%    300 dpi in resolution and 500 dpi minimum if line art images.  RGB, CMYK, and Grayscale are all 
%    acceptable. Halftones should be high contrast with sharp detail, because some loss of detail and 
%    contrast is inevitable in the production process. Figures should be 10-20 cm in width and 1-25 cm 
%    in height. Graph axes must be exactly perpendicular and all lines of equal density.  Label 
%    multiple figure parts with A, B, etc. in bolded type, and use Arrows and numbers to draw attention 
%    to areas you want to highlight. Legends should start with a brief title and should be a 
%    self-contained description of the content of the figure that provides enough detail to fully 
%    understand the data presented. All conventional symbols used to indicate figure data points are 
%    available for typesetting; unconventional symbols should not be used. Italicize all mathematical 
%    variables (both in the figure legend and figure) , genotypes, and additional symbols that 
%    are normally italicized.  
%}%
%\label{fig:spectrum}
%\end{figure}
%
%\subsection*{Sample Video}
%
%Figure \ref{video:spectrum} shows how to include a video in your manuscript.
%
%\begin{figure}[htbp]
%\renewcommand{\familydefault}{\sfdefault}\normalfont
%\centering
%\includegraphics[width=\linewidth]{images/example-figure-g3}
%\caption{Example movie (the figure file above is used as a placeholder for this example). G3 supports video and movie 
%         files that can be linked from any portion of the article - including the abstract. Acceptable formats include 
%         .asf, avi, .wav, and all types of Windows Media files.   
%}%
%
%\label{video:spectrum}
%\end{figure}
%
%
%\subsection*{Sample Table}
%
%Table \ref{tab:shape-functions} shows an example table. Avoid shading, color type, line drawings, graphics, or 
%                                other illustrations within tables. Use tables for data only; present drawings, graphics, 
%                                and illustrations as separate figures. Histograms should not be used to present data 
%                                that can be captured easily in text or small tables, as they take up much more space.  
%
%Tables numbers are given in Arabic numerals. Tables should not be numbered 1A, 1B, etc., but if necessary, 
%interior parts of the table can be labeled A, B, etc. for easy reference in the text.  


%\begin{table*}[htbp]
%\renewcommand{\familydefault}{\sfdefault}\normalfont
%\centering
%\caption{\bf Students and their grades}
%\begin{tableminipage}{\textwidth}
%\begin{tabularx}{\textwidth}{XXXX}
%\hline
%\header Student & Grade\footnote{This is an example of a footnote in a table. Lowercase, superscript italic letters (a, b, c, etc.) are used by default. You can also use *, **, and *** to indicate conventional levels of statistical significance, explained below the table.} & Rank & Notes \\
%\hline
%Alice & 82\% & 1 & Performed very well.\\
%Bob & 65\% & 3 & Not up to his usual standard.\\
%Charlie & 73\% & 2 & A good attempt.\\
%\hline
%\end{tabularx}
%  \label{tab:shape-functions}
%\end{tableminipage}
%\end{table*}

%\section*{Sample Equation}
%
%Let $X_1, X_2, \ldots, X_n$ be a sequence of independent and identically distributed random variables with $\text{E}[X_i] = \mu$ and $\text{Var}[X_i] = \sigma^2 < \infty$, and let
%\begin{equation}
%S_n = \frac{X_1 + X_2 + \cdots + X_n}{n}
%      = \frac{1}{n}\sum_{i}^{n} X_i
%\label{eq:refname1}
%\end{equation}
%denote their mean. Then as $n$ approaches infinity, the random variables $\sqrt{n}(S_n - \mu)$ converge in distribution to a normal $\mathcal{N}(0, \sigma^2)$.

\bibliography{bibliography}

\end{document}
