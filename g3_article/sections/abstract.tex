\begin{abstract}

Reduced costs for DNA marker technology coupled has generated a huge amount of
molecular data and greatly increased the options available to characterize lines 
in a breeding program. Concurrently, the field of machine learning has experienced
a resurgence of research into techniques to detect or "learn" patterns in noisy
data in a variety of technical applications. Here, we apply so called "deep learning"
techniques from current machine learning research related to neural networks to five 
genomic selection and phenotypic selection problems using published reference datasets. 
We compare the results of these algorithms to a selection of bayesian and linear 
regression techniques commonly employed today. TODO: FINDINGS/RESULTS/CONCLUSIONS.


%The abstract should be written for people who may not read the 
%entire paper, so it must stand on its own.  The impression it makes usually determines 
%whether the reader will go on to read the article, so the abstract must be 
%engaging, clear, and concise.  In addition, the abstract may be the only part of the 
%article that is indexed in databases, so it must accurately reflect the 
%content of the article. A well-written abstract is the  most effective way to reach intended 
%readers, leading to more robust search, retrieval, and usage of the article. 
%
%Please see additional guidelines notes on preparing your abstract below.

%\begin{itemize}
%\item provide a synopsis of the entire article;
%\item begin with the broad context of the study, followed by specific background for the study;
%\item describe the purpose, methods and procedures, core findings and results, and conclusions of the study;
%\item emphasize new or important aspects of the research;
%\item engage the broad readership of G3 and be understandable to a diverse audience (avoid using jargon);
%\item be a single paragraph of less than 250 words;
%\item contain the full name of the organism studied;
%\item NOT contain citations or abbreviations.
%\end{itemize}

\end{abstract}

