

Reduced costs for DNA marker technology has generated a huge amount of
molecular data and made it economically feasible to generate dense genome-wide
marker maps of lines in a breeding program. Increased data density and volume
has driven an exploration of tools and techniques to analyze these data for
cultivar improvement. Research into data science theory and application has experienced
a resurgence of research into techniques to detect or "learn" patterns in noisy
data in a variety of technical applications. Several variants of machine learning have been proposed
for analyzing large DNA marker data sets to aid in phenotype prediction and genomic selection.
Here, we apply deep learning techniques from machine learning research to six 
phenotypic prediction tasks using published reference datasets. Because regularization
frequently improves neural network prediction accuracy, we included regularization
methods in the neural network models.
The neural network models are compared to a selection of Bayesian and 
linear regression techniques commonly employed for phenotypic prediction and
genomic selection. Applying regularization improved neural network
accuracy. On three of the phenotype prediction tasks, regularized neural networks 
were the most accurate of the models evaluated. Surprisingly, for these data sets
the depth of the network architecture did not affect the accuracy of the trained 
model. We also find that concerns about the computer processing
time needed to train neural network models to perform well in genomic prediction 
tasks may not apply when Graphics Processing Units are used for model training.


%The abstract should be written for people who may not read the 
%entire paper, so it must stand on its own.  The impression it makes usually determines 
%whether the reader will go on to read the article, so the abstract must be 
%engaging, clear, and concise.  In addition, the abstract may be the only part of the 
%article that is indexed in databases, so it must accurately reflect the 
%content of the article. A well-written abstract is the  most effective way to reach intended 
%readers, leading to more robust search, retrieval, and usage of the article. 
%
%Please see additional guidelines notes on preparing your abstract below.

%\begin{itemize}
%\item provide a synopsis of the entire article;
%\item begin with the broad context of the study, followed by specific background for the study;
%\item describe the purpose, methods and procedures, core findings and results, and conclusions of the study;
%\item emphasize new or important aspects of the research;
%\item engage the broad readership of G3 and be understandable to a diverse audience (avoid using jargon);
%\item be a single paragraph of less than 250 words;
%\item contain the full name of the organism studied;
%\item NOT contain citations or abbreviations.
%\end{itemize}

