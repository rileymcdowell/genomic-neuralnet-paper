\subsection*{Prediction Pipeline}

%Three benchmark datasets of paired phenotypic and genotypic marker data were analyzed 
%using a collection of regression techniques. Three species were selected
%as benchmark species. Benchmark traits included both high and low heritability 
%traits with simple and complex genetic architectures. 

Each set of paired phenotypic and genotypic measurements were divided into 
five partitions by drawing paired measurements without replacement. Each prediction model
was trained on four partitions of the data and used to predict the remaining held-out partition. 
For statistical models with tunable hyperparameters, a grid-search of the parameter 
space was conducted and the model predicted out outcome will all sets of parameters in the grid.
The prediction accuracy as well as any model parameters were recorded for the partition. The models 
were then reset and the process was repeated holding out a different partition. This produced 
five estimates of the prediction accuracy of each model. The datasets were then shuffled 
and the pipeline was re-run to produce a total of ten prediction accuracy estimates for each model. 

\subsection{Neural Network Implementation and Training}

All neural network models were initialized using Glorot weight initialization \citep{glorot2010} 
and trained for 12,100 epochs of four batches each. Backpropagation was performed using the 
Nadam backpropagation algorithm with standard parameters \citep{dozat2015}. 
The first 100 epochs were executed on separate networks, and the network with the lower 
in-sample prediction accuracy was allowed to complete the remaining 12,000 epochs, 
reducing the incidence of low accuracy due to poor initial network weights.
The network learning rate was reduced by a factor of four after 10,100 epochs and reduced by a further
factor of four after 11,600 epochs as a simple form of learning rate decay.

The full analysis pipeline, including formatted datasets, quality control processes, 
implementations of each model, and final hyperparameter settings in file databases 
are available in versioned files in a public repository on github.com \citep{mcdowell2016}.

\subsection*{Statistical Analysis} 

Least squares, ridge, LASSO, and elastic net regression were applied as examples of
linear regression with and without normalization penalties. A Bayesian ridge regression 
model was included as an example of a model with an enforced Bayesian prior distribution
on coefficient values. These five regression techniques were available in the 
scikit-learn python package version 0.17.1 \citep{scikit-learn}.

A single parameterized neural network with dropout and weight-decay parameters was built using 
the Keras modular neural network library \citep{chollet2015}. The neural network results presented in this
manuscript were trained and evaluated using the default Theano backend \citep{team2016}. 
Network training was conducted on a combination of a single NVIDIA GTX 680 graphics card with 4GB 
of RAM and a cluster of NVIDIA GRID K520 graphics cards with 4GB of RAM each. Both types GPUs used the 
CUDA 7.5 toolkit \citep{nickolls2008}. A network with no regularization,
as well as with either one or both of the weight-decay and dropout regularization parameters
supplied was trained on each dataset. 

A summary of all regression methods is presented in Table \ref{tab:regression-methods}.

\ifdefined\showtablesandfigures
% Benchmark Datasets table.

\begin{table*}[htbp]
    \renewcommand{\familydefault}{\sfdefault}\normalfont
    \centering
    \caption{\bf Regression Methods}

    \begin{tableminipage}{\textwidth}

        \begin{tabularx}{\textwidth}{ X m{5em} m{10em} }
        \hline
        \header Method & Abbreviation & Library \\
        \hline
        Ordinary Least Squares Regression            & OLS   & scikit-learn   \\
        Ridge Regression                             & RR    & scikit-learn   \\
        LASSO Regression                             & LASSO & scikit-learn   \\
        Elastic Net Regression                       & EN    & scikit-learn   \\ 
        Baysian Ridge Regression                     & BRR   & scikit-learn   \\
        Unregrularized Neural Network                & N     & Keras (Theano) \\
        Neural Network with Weight Decay             & NWD   & Keras (Theano) \\
        Neural Network with Dropout                  & NDO   & Keras (Theano) \\
        Neural Network with Weight Decay and Dropout & NWDDO & Keras (Theano) \\
        \hline
        \end{tabularx}


        \label{tab:regression-methods}
        \footnotesize  
    \end{tableminipage}
\end{table*}
 % Label = tab:regression-methods.
\fi

Each regression technique was trained and evaluated on each phenotypic trait in all 
ten (5*2) cross-validation partitions of each dataset. The correlation of the 
predicted and actual phenotypic values was taken for each held-out partition, 
and the average and standard deviation of the resulting correlation coefficients 
for each regression technique were reported.

\subsection*{Benchmark Datasets}

Arabidopsis, maize, and wheat datasets were collected from the author's web pages
or the supplementary information published with their respective papers \citep{loudet2002, crossa2010, thavamanikumar2015}.
Species, authorship, marker, and sample information is summarized in Table \ref{tab:benchmark-datasets}.

\ifdefined\showtablesandfigures
% Benchmark Datasets table.

\begin{table*}[htbp]
\renewcommand{\familydefault}{\sfdefault}\normalfont
\centering
\caption{\bf Benchmark Datasets}
\begin{tableminipage}{\textwidth}


\newcommand{\etal} {\textit{et al.}}

\newcommand{\loudet}        {\multirow{2}{*}{\parbox{4.0em}{Loudet \etal ~2002}}}
%\newcommand{\resende}       {\multirow{2}{*}{\parbox{4.5em}{Resende \etal ~2012}}}
\newcommand{\crossa}        {\multirow{2}{*}{\parbox{4.0em}{Crossa \etal ~2010}}}
%\newcommand{\cleveland}     {\multirow{2}{*}{\parbox{4.5em}{Cleveland \etal ~2012}}}
\newcommand{\thavamanikumar}{\multirow{2}{*}{\parbox{4.5em}{Thavamanikumar \etal ~2015}}}
\newcommand{\phenom}        {\multirow{2}{*}{\parbox{4.0em}{Measure-ment}}}
\newcommand{\blupm}         {\multirow{2}{*}{BLUP}}


\newcommand{\arabidopsis}   {\multirow{2}{*}{Arabidopsis}}
\newcommand{\loblolly}      {\multirow{2}{*}{Loblolly Pine}}
\newcommand{\maize}         {\multirow{2}{*}{Maize}}
\newcommand{\pig}           {\multirow{2}{*}{Pig}}
\newcommand{\wheat}         {\multirow{2}{*}{Wheat}}


\begin{tabularx}{\textwidth}{ m{7em} m{5em} m{10em} m{4em} m{4em} m{5em} }
\hline
\header Author & Species & Trait & Markers\footnote{Markers with greater than 20\% missing calls were filtered from analysis.} & Samples\footnote{Samples with missing phenotypic measurements or greater than 50\% missing marker calls were filtered from analysis.} & Dependent Variable \\
\hline
\loudet         & \arabidopsis & Days to flowering w/ short day length     & 69     & 415   & \phenom \\
                &              & Low resource dry matter accumulation      & 69     & 415   &         \\
%\hline
%\resende        & \loblolly    & Crown width age 6 yrs                     & 4,700  & 861   & \phenom \\
%                &              & Wood lignin age 4 yrs                     & 4,698  & 910   &         \\
\hline
\crossa         & \maize       & Well-watered grain yield                  & 1,135  & 264   & \phenom \\
                &              & Female flowering date                     & 1,148  & 284   &         \\
%\hline
%\cleveland      & \pig         & Trait 1 (low heritabilty)                 & 52,843 & 2,804 & \phenom \\
%                &              & Trait 5 (high heritability)               & 52,843 & 3,184 &         \\
\hline
\thavamanikumar & \wheat       & Time to young microspore                  & 797    & 324   & \blupm  \\
                &              & Spike grain number                        & 797    & 324   &         \\
\hline
\end{tabularx}

\label{tab:benchmark-datasets}
\footnotesize  
\end{tableminipage}
\end{table*}
 % Label = tab:benchmark-datasets.
\fi

Marker calls were scaled to the range $[0,1]$ for all SNP information. If more than 20\% of
marker calls were missing for a sample, the sample was discarded. If fewer than 20\% were missing,
the average value for that marker was imputed for the missing values with one exception: if data were 
published with a marker imputation technique already applied, no further imputation was attempted.

Individuals without phenotypic measurements were discarded from further analysis. A combination of phenotypic 
measurements and deregressed breeding values were predicted, depending on which measure was provided by the authors.

In addition to their original publications, copies of each of the benchmark datasets as well 
as modified versions formatted for compatibility with the analyses presented in 
this paper are available on github \citep{mcdowell2016}. The analysis framework 
is extensible and can be utilized to evaluate arbitrary datasets and predictive models
with minimal coding requirements provided the statistical models are implemented
in the python 2.7 programming language.

